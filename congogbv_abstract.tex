\begin{abstract}
The high incidence rates of Sexual and Gender-based Violence in the Democratic Republic of the Congo presents a large human rights problem. Despite the attention this topic has been given, much is unclear about the drivers of SGBV, as accurate data on the subject is difficult to collect. In this paper,I explore the characteristics of the victims SGBV to uncover the dynamics and potential drivers of SGBV. I focus on conflict and female empowerment. In order to avoid social desirability bias and obtain reliable data on this incidence, I conducted a list experiment. I combine the data from this list experiment with rich data on respondents' households, which allows me to separate conflict into recent conflict and historic conflict; and have detailed information on the position of women within their households.

I find that female victims of SGBV are likely to be married to higher-status men, have low intra-household bargaining power, and have been exposed to violent conflict to the extent where they have lost family or household members before 2012 (two years before the list experiment). These findings coincide with the view of a long-lasting impact of conflict on SGBV rates, mainly through Intimate Partner Violence (IPV). This means that human rights abuses  persist long after the end of conflict. Strong measures to structurally improve the position of women in the household and society as a whole are required to address this.
\end{abstract}