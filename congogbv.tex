\documentclass[11pt,a4paper]{scrartcl} %scrartcl from KOMA-script uses sans headers and looks less latexy than article
%\usepackage[doublespacing]{setspace}
%\usepackage[capposition=top]{floatrow} %for notes under pics
%\usepackage[nomarkers]{endfloat}
\usepackage{natbib}
\usepackage[utf8]{inputenc}
%\usepackage{amsmath}
%\usepackage{amsfonts}
%\usepackage{amssymb}
\usepackage[left=2.5cm,right=2.5cm,top=3cm,bottom=3cm]{geometry}
\usepackage[pdftex, hidelinks]{hyperref}
\usepackage{graphicx}
\usepackage{booktabs}
\usepackage[blocks]{authblk} %for affil
\usepackage{url}
\usepackage{tabularx}
\usepackage{csvsimple}
\usepackage[nomessages]{fp}
\usepackage{xcolor}
\usepackage{enumitem}
\setlist{noitemsep} % \setlist{nosep} to reove separation around lists as well


%citation cheat sheet:
	%\ cite{key}				Jones et al. (1990)
	% \citet{key1,key2}			Jones et al. (1990); Ladze (1986)
	% \citet{key}				Jones et al. (1990)
	% \citet*{key}				Jones, Baker, and Smith (1990)
	% \citep{key}				(Jones et al. 1990)
	% \citep{key1,key2}			(Jones et al. 1990; Ladze 1986)
	% \citep*{key}				(Jones, Baker, and Smith 1990)
	% \citep[p.~99]{key}		(Jones et al., 1990, p. 99)
	% \citep[e.g.][]{key}		(e.g. Jones et al., 1990)
	% \citep[e.g.][p.~99]{key}	(e.g. Jones et al., 1990, p. 99)
	% \citeauthor{key}			Jones et al.
	% \citeauthor*{key}			Jones, Baker, and Smith
	% \citeyear{key}			1990
	% \citealt{key} 			Jones et al. 1990


\newcommand{\figureloc}{C:/Users/Koen/Dropbox/PhD/Papers/CongoGBV/Figures}
\newcommand{\tableloc}{C:/Users/Koen/Dropbox/PhD/Papers/CongoGBV/Tables}
\newcommand{\bibloc}{C:/Users/Koen/Dropbox/Literatuur/Mendeley/Bibtex/CongoGBV}





\begin{document}

\author{Koen Leuveld}

\affil{Development Economics Group, Wageningen University \\
\href{mailto:koen.leuveld@wur.nl}{koen.leuveld@wur.nl}}



\title{Sexual Violence, conflict and intra-household relations}
\subtitle{Exploratory evidence from a list experiment in Eastern DR Congo} %needs scrartcl

\maketitle

\begin{center}
\textcolor{red}{\Large VERY EARLY DRAFT!!!!}
\end{center}
%\begin{abstract}
%Abstract to come
%\end{abstract}


%%%%%%%%%%%%%%%%%%%%%%%%%%
\section*{Introduction}
%%%%%%%%%%%%%%%%%%%%%%%%%

\paragraph{}
%Hook
The high incidence of Sexual and Gender Based Violence (SGBV) in the Congo has been a prominent area of attention for the international development community over the past decades. Often, this high incidence of SGBV has been linked to the various violent conflicts the country has seen since the 1990s. In the policy domain, attention for the link between conflict in general and SGBV increased at the end of 1990s, when rape was used strategically by armed actors during armed conflicts in Rwanda and Bosnia  \citep{Kirby2015}. In Congo in particular, the issue has been on the agenda since 2002 when the NGO Human Rights Watch published a report on the topic \citep{HRW2002}. Since then, the issue has received attention by the UN, governments, NGOs and even celebrities \citep{Baaz2013}. This attention is understandable, as even outside of conflict settings, the psychological, social and economic costs of SGBV are staggering \citep{Post2002,Peterson2018}. To use this a weapon defies the imagination of Western audiences. Consequently, Congo to nearly become synonymous with rape, even called the ``rape capital of the world''. Tremendous international efforts have been made to implement or support projects to assist the victims of SGBV. The 2018 Nobel Peace prize was awarded to Dr. Denis Mukwege, for his work on victims of SGBV at Bukavu's Panzi Hospital.
 
\paragraph{}
%research question
To adequately address this issue, data on the victims of SGBV is of crucial importance. In this paper, I explore the characteristics of the victims SGBV.In particular, I consider conflict victimization, intra-household dynamics, and poverty as potential characteristics.

\paragraph{}
%Antecedents
While estimates for the incidence of SGBV in DR Congo vary, they are invariably high: ranging from 15\% to 40\% \citep{Johnson2010,Peterman2011}. The conflict that has persisted in the country for the past decades has been frequently cited as the most important driver of SGBV. This is particularly true for policy circles, where the framing of SGBV in Congo as ``weapon of war'' is popular \citep{Baaz2013,Kirby2015}. There is empirical evidence that backs up this framing. \cite{Johnson2010} carried out a large-scale survey in Eastern DRC to investigate incidence and perpetrators of GBV. They find that the majority of sexual violence reported by their respondents was conflict-related; of female victims of sexual violence, 74.3\% reported being victim to conflict-related sexual violence. It is therefore not surprising that the topic of GBV in Congo has often been analysed within the context of violence \citep{Baaz2013}. Likewise, the aspect of the conflict that has received the most world-wide media attention has been GBV \citep{Autesserre2012a}.  
%Bouta, T., G. Frerks, and I. Bannon (2005) Gender, Conflict and Development. The World Bank, Washington, DC
\paragraph{}
This view of the central role of conflict in sexual violence has come under increasing scrutiny. It has been argued that this focus on the relationship between sexual violence and conflict has been counter-productive, as it has distracted attention from other pressing problems the country faces \citep{Autesserre2012a, Hilhorst2018,Porter2019}. Moreover, it risks missing the civilian perpetrators of SGBV, which are at least equally important as perpetrators from armed groups. There is empirical evidence for this position too. Based on DHS data, \cite{Peterman2011} find that Intimate Partner Violence is a large contributor to the incidence of sexual violence in Congo. \cite{Hilhorst2018} argues that the local NGO community in Eastern Congo is well aware of this, and uses the increased funds related to the massive attention generated by the ``weapon of war'' account of SGBV in Congo to fund projects to improve female empowerment in Congo and in this way prevent SGBV.  However, there is a long way to go, and women still hold a precarious position in Congolese society. For example, views that men have the right to physically abuse their wife if she is disobedient are broadly held \citep{Quattrochi2019}.
%Tia Palermo and Amber Peterman, ‘Undercounting, overcounting and the longevity of flawed estimates: statistics on sexual violence in conflict’, Bulletin of the World Health Organization, no. 89, 2011, pp. 924–5;
\paragraph{}
A large problem underlying the analysis of drivers and dynamics of SGBV is the paucity of reliable data. This exacerbated by the fact that the empirical studies that do exist, contradict each other. While one study finds conflict-related perpetrators are responsible for the majority of cases of SGBV \citep{Johnson2010}, another finds intimate partners as the most common culprits \citep{Peterman2011}. \citet{Stark2017} provide a possible explanation for this discrepancy. In a study where 87 women and girls in South Kivu were interviewed using Audio Assisted Self-Interviews (ACASI) techniques,they find that 14\% of respondents reported having been victim of sexual coercion. Half of these were perpetrated by the respondents' husbands or boyfriends. Crucially, in complementary group discussions, respondents did not bring up intimate partners at all. This points at the difficulties of collecting accurate data about GBV, and the importance of data collection methodologies. 

\paragraph{}
%Contribution
This paper contributes to the empirical evidence base on the incidence of GBV in Eastern Congo by using a list experiment to obtain reliable estimates of the incidence of SGBV. List experiments remove the bias in answering sensitive question, by removing the possibility of the researcher (or anyone else) to view individual answers to the sensitive topic. However, group-level analysis is still possible, to provide estimates of correlates of GBV. Combined with rich data from a household survey and behavioural experiments, collected in 2012 and 2014, this approach allows for a rich characterization of victims of GBV. As such a characterization is lacking thusfar, this data can be of tremendous usefulness in addressing and preventing GBV.
%perhaps note that tia finds no predictors for GBV.

\paragraph{}
%Road Map
I find no correlation between victimization of GBV and victimization of the conflict. I do find a strong correlation between victimization of GBV and intra-household dynamics: the result from a bargaining game, and the pre-marriage relative status of the wife and husband. This early draft of the paper is structured as follows: first I describe the research setting and data. I then present the results from various analyses on the list experiments conducted. In the concluding remarks, I contextualize the findings and present policy implications.

%%%%%%%%%%%%%%%%%%%%%%%%%%%%%%%%
\section*{Background and data}
%%%%%%%%%%%%%%%%%%%%%%%%%%%%%%%%
\paragraph{}
%to do in this section:
%add map.
%answer data questions
%add distributions of balls
The main source of data for this study is a household survey that was undertaken in 2014 as part of the evaluation of Dutch development aid in Eastern Congo. This evaluation concerned projects ran by three NGOs in the territories of Fizi and Uvira in South Kivu province. The household questionnaire contained a final module that: (i) collected detailed information on gender attitudes; (ii) contained a risk barganing game; and (iii) contained a list experiment designed to elicit the indicence of SGBV among female respondents. Where possible, this gender module was administered to both the head of the household and the spouse of the household head (in the vast majority of the cases, the husband is considered the head, but it was left open to the respondent to indicate the head). 


\begin{table}[htbp]
	\centering
	\caption{Gender module sample make up}
	\label{tab:bargsample}
	\begin{tabular}{l c c c c c}
		\toprule
		\input{\tableloc/tabs.tex}
		\bottomrule
	\end{tabular}
\end{table}

\paragraph{}
Table \ref{tab:bargsample} displays how the sample is built up. In total, there were 889 respondents to the survey. In 593 household, the wife (or the single female head, for convenience I will refer to both as wife) consented to responding to the gender module. In 1 household, the wife refused; in 5, the wife was absent during the interview, and in 290 households the head of the household had no wife. In 470 households, the husbands consented to the module, 6 refused, 255 husbands were absent, and in 158 households the head of the household was an unmarried woman. For 184 households, we have both husband and wife responding to the module. Efforts to increase this number, by tracking down absent household heads, were constricted by the limited time field teams had in each community, due to the security situation at the time of field work. \textcolor{red}{HOW DOES THIS AFFECT MY ESTIMATES, AND HOW CAN I MITIGATE THIS?}

\paragraph{}
For the list experiment, the respondents were randomly divided into two groups. I follow \cite{Imai2011} in calling these groups Treatment and Control. Each group was presented with a number of issues that women can realistically face in Eastern Congo. They then indicated how many of the issues they themselves had faced. The first group (Control) was presented with the following issues:
\begin{itemize}
	\item Lack of food;
	\item Lack of money;
	\item Theft; and,
	\item Sterility.
\end{itemize}

\paragraph{}
 Women in the second group (Treatment) were presented with the same four items, but a fifth item was added: Sexual Violence. The questionnaire was field tested prior to field work to ensure that respondents understood these concepts. While using a list experiment means I cannot know who among the respondents was the victim of GBV, I can compare the means of the number of issues between the groups to arrive at an estimate of SGBV. By comparing means between sub-samples, and by using more sophisticated methods proposed by \citet{Imai2011} (and implemented by \cite{Tsai2019} in Stata), I can find the correlates of SGBV. 

 \paragraph{}
 The risk bargaining game in the module was modified from \cite{Martinsson2009}. In the game, couples choose between a set of six risky lotteries, based on \cite{Eckel2002}. The lotteries presented range from fairly low-risk ones -- where low and high pay-out are nearly equal -- to high-risk one -- where there is a large difference between high and low pay-outs (see Table \ref{tab:bargaining} for details of the lotteries). The respondents first choose individually (without knowing their partner's choice), and then jointly. By comparing the couple decision with the individual decision, I obtain an indicator for barganing power. The difference between the procedure used by \cite{Martinsson2009} and the one here, is that they use a risk experiment based on \cite{Holt2002}. However, this is a more complicated experiment compared to Eckel and Grossman. This added complication may cause participants to not fully understand the procedure, leading to poor results \citep{Dave2010a}. Given the low numeracy of the subjects, I implemented the simpler of the two experiments.

\begin{table}
	\centering
	\caption{Bargaining game lotteries}
	\label{tab:bargaining}
	\begin{tabular}{l c c c c}
	\toprule
	\# & Low & High & Expected & Risk aversion \\
	\hline
	1 & 4,000 CDF & 4,000 CDF & 4,000 CDF & Extremely risk-averse \\
	2 & 3,600 CDF & 4,800 CDF & 4,200 CDF & Extremely risk-averse \\
	3 & 3,200 CDF & 5,600 CDF & 4,400 CDF & Moderately risk-averse \\
	4 & 2,800 CDF & 6,400 CDF & 4,600 CDF & Moderately risk-averse \\
	5 & 2,400 CDF & 7,200 CDF & 4,800 CDF & Risk-neutral \\
	6 & 1,400 CDF & 8,200 CDF & 4,800 CDF & Risk-loving \\ \bottomrule
	\end{tabular}
\end{table}

\paragraph{}
Conflict victimization data is not available for the 2014 round of the survey. However, for 479, conflict data is available from the 2012 round of data collection. Since the majority of conclict events happened before this time, I used this data as a proxy for conflict exposure.

\paragraph{}
Table \ref{tab:balance} presents descriptive statistics of the sample. Our sample consists of 589 women. For 383 women, I have detailed information on their marriage. These women are married to head of the household. One question asked was whether the family of the wife or the husband had more land at the time of marriage. Given the importance of agriculture in the area, this amount of land is a good proxy for status. In 38\% of the cases, the husband's family had more land, in 24\% of the cases the wife's family did. For the bargaining game, I have 182 respondents. This number is lower than the 383 married couples, because both spouses needed to be present in the two days research teams spent in each community.If they were present, both spouses would need to consent to the risk game. In 40\% of the cases, the couple's bargained decision was closest to the husbands. In 27\% of the cases, the couple's decision was closest to the wife's. In the remainder, the final decision was equally close to both. Conflict victimization is high in the sample. 80\% of the household indicate having lost property due to the conflict; 53\% report having lost a family or household member. The women in the sample, on average, contribute 20\% of the cash income in their household, and 24\% of the in-kind income. About 61\% of the households have a tin roof (with the vast majority of the remainder owning simple thatch roofs), and 49\% own any kind of livestock.

\paragraph{}
%is hh_id 1 determined randomly? If not, perhaps exlude it?
The treatment and control group are not perfectly balanced across some of the variables presented in table \ref{tab:balance}. As randomization was done at the time of data collection (based on randomly assigned ID codes). An F-test on the differences between treatment and control being jointly equal to zero fails to reject the null-hypothesis that they are equal (p=0.20). However, care should be taken in interpreting the differences in the variables with differences: results being closer to the women's choice, and the cash income of the household. 

\input{\tableloc/balance.tex}

%some examples on how to interpret list experiments:
%\cite{Meng2017} run various types of models. Consistently refer to mean difference models, and Maximum Likelihood models

%\cite{Blair2014}: extremely detailed stuff. Thread carefully!
%\cite{Frye2017} simple difference in means model.

%%%%%%%%%%%%%%%%%%%%%%%%%%%%%%%%
\section*{Results}
%%%%%%%%%%%%%%%%%%%%%%%%%%%%%%%%
%command to get incidences from csv file
\newcommand{\incid}[2]{\csvreader[filter strcmp={\key}{#1}]{\tableloc/incidence.csv}{key=\key,#2=\inc}{\inc}}

\begin{figure}[hp]
  \includegraphics[width=0.6\linewidth]{\figureloc/meancompare_overall.png}
  \caption{Comparison of means of issues faced: treatment vs. control.}
  \label{fig:meancompare_overall}
\end{figure}

\paragraph{}
First, I compare the number of issues faced by the two groups of women. The difference between the group who were presented with only four issues (the control group) and the group who were presented four issues plus GBV (the treatment group) is my estimate of the incidence of GBV. The average number of issues reported by the control group is \incid{overall1}{mean0}, while the number if issues reported by the treatment group is \incid{overall1}{mean1} (see Figure \ref{fig:meancompare_overall}). The difference of \incid{overall1}{incidence} implies that the incidence of GBV is \incid{overall1}{incidence_pct}\% in this sample. The p-value for a t-test on this difference is \incid{overall1}{p}. This incidence is slightly higher than some estimates \citep{Quattrochi2019}.

\subsection*{Intra-household dynamics}
\begin{figure}[hp]
  \includegraphics[width=0.6\linewidth]{\figureloc/meancompare_mar1.png}
  \caption{Comparison of means of issues faced by pre-marriage status.}
  \label{fig:meancompare_mar1}
\end{figure}

\paragraph{}
I then compare the number of issues faced in several sub-groups. First, I focus on marriage dynamics. I compare women across the relative status of the partners at the time of marriage. For this, I use the land held by the partners' families as proxy for status (figure \ref{fig:meancompare_mar1}). Given the importance of agriculture, and conversely land, in the study area, this is a reasonably proxy. In the group where the wife had more status before the marriage (n=\incid{statpar1}{n}), the average number of issues was \incid{statpar1}{mean0} for the control group, and \incid{statpar1}{mean1} for the treatment group. The difference in means implies a rate of GBV of \incid{statpar1}{incidence_pct} (p-value of a t-test on the means = \incid{statpar1}{p}). In the group where the partners were of equal standing (n=\incid{statpar2}{n}), the average number of issues was \incid{statpar2}{mean0} for the control group, and \incid{statpar2}{mean1} for the treatment group. This implies a rate of GBV of \incid{statpar2}{incidence_pct} (p = \incid{statpar2}{p}). In the group where the husband had more status (n=\incid{statpar3}{n}), the average number of issues was \incid{statpar3}{mean0} for the control group, and \incid{statpar3}{mean1} for the treatment group. This implies a rate of GBV of \incid{statpar3}{incidence_pct} (p = \incid{statpar3}{p}).

\paragraph{}
The second intra-household aspect I explore is derived from the results of a bargaining game played with couples. Couples first took a decision individually, and then jointly. I then create three groups, based on whether the joint decision is closer to the husband's decision, to the wife's, or if the distance is equal. See figure \ref{fig:meancompare_mar2}. In the group where the couple decision was closest to the wife's decision (n=\incid{bargresult1}{n}), the average number of issues was \incid{bargresult1}{mean0} for the control group, and \incid{bargresult1}{mean1} for the treatment group. This implies a rate of GBV of \incid{bargresult1}{incidence_pct}\% (p = \incid{bargresult1}{p}). In the group where the couple decision was equally close to the husband and wife (n=\incid{bargresult2}{n}), the average number of issues was \incid{bargresult2}{mean0} for the control group, and \incid{bargresult2}{mean1} for the treatment group. This implies a rate of GBV of \incid{bargresult2}{incidence_pct}\% (p = \incid{bargresult2}{p}). In the group where the couple decision was closest to the husband's (n=\incid{bargresult3}{n}), the average number of issues was \incid{bargresult3}{mean0} for the control group, and \incid{bargresult3}{mean1} for the treatment group. This implies a rate of GBV of \incid{bargresult3}{incidence_pct}\% (p = \incid{bargresult3}{p}).

\begin{figure}[hp]
  \includegraphics[width=0.6\linewidth]{\figureloc/meancompare_mar2.png}
  \caption{Comparison of means of issues faced: forced marriage vs. other marriage types.}
  \label{fig:meancompare_mar2}
\end{figure}

\paragraph{}
The final intra-household variable I use to compare women, is their relative contribution to the household's cash income (see figure \ref{fig:meancompare_mar3}). The household questionnaire collected data on all activities of household members, and what the contribution of these activities were to household income. I sum this for all activities for the women in the sample, and compare those who contribute more than 50\% of cash income with those who do not. In the group where the woman does not contribute more than 50\% of cash income (n=\incid{contribcashyn0}{n}), the average number of issues was \incid{contribcashyn0}{mean0} for the control group, and \incid{contribcashyn0}{mean1} for the treatment group. This implies a rate of GBV of \incid{contribcashyn0}{incidence_pct}\% (p = \incid{contribcashyn0}{p}). In the group where the woman does contribute more than 50\% of cash income (n=\incid{contribcashyn1}{n}), the average number of issues was \incid{contribcashyn1}{mean0} for the control group, and \incid{contribcashyn1}{mean1} for the treatment group. This implies a rate of GBV of \incid{contribcashyn1}{incidence_pct}\% (p = \incid{contribcashyn1}{p}). 

\begin{figure}[hp]
  \includegraphics[width=0.6\linewidth]{\figureloc/meancompare_mar3.png}
  \caption{Comparison of means of issues faced across contribution to cash income.}
  \label{fig:meancompare_mar3}
\end{figure}

\subsection*{Conflict}
I then compare the conflict history of the respondents' households, to assess whether conflict victimization correlates with the incidence of GBV. I use data collected from an earlier round of the study that collected detailed conflict exposure data. First, I compare respondents who live in households who have suffered loss of (or damage to) property, including agricultural fields, due to conflict. See \ref{fig:meancompare_conf1}. In the non-victimized group (n=\incid{victimproplost0}{n}), the average number of issues was \incid{victimproplost0}{mean0} for the control group, and \incid{victimproplost0}{mean1} for the treatment group. This implies a rate of GBV of \incid{victimproplost0}{incidence_pct}\% (p = \incid{victimproplost0}{p}). For the group that has suffered loss of property due to the conflict (n=\incid{victimproplost1}{n}), the average number of issues was \incid{victimproplost1}{mean0} for the control group, and \incid{victimproplost1}{mean1} for the treatment group. This implies a rate of GBV of \incid{victimproplost1}{incidence_pct}\% (p = \incid{victimproplost1}{p}).

\begin{figure}[hp]
  \includegraphics[width=0.6\linewidth]{\figureloc/meancompare_conf1.png}
  \caption{Comparison of means of issues faced across conflict exposure.}
  \label{fig:meancompare_conf1}
\end{figure}

The second conflict indicator I examine, is whether the respondent's household has lost any household members or family as a consequence of the conflict.See \ref{fig:meancompare_conf2}. In the non-victimized group (n=\incid{victimfamlost0}{n}), the average number of issues was \incid{victimfamlost0}{mean0} for the control group, and \incid{victimfamlost0}{mean1} for the treatment group. This implies a rate of GBV of \incid{victimfamlost0}{incidence_pct}\% (p = \incid{victimfamlost0}{p}). For the group that has suffered loss of property due to the conflict (n=\incid{victimfamlost1}{n}), the average number of issues was \incid{victimfamlost1}{mean0} for the control group, and \incid{victimfamlost1}{mean1} for the treatment group. This implies a rate of GBV of \incid{victimfamlost1}{incidence_pct}\% (p = \incid{victimfamlost1}{p}).

\begin{figure}[hp]
  \includegraphics[width=0.6\linewidth]{\figureloc/meancompare_conf2.png}
  \caption{Comparison of means of issues faced across conflict exposure.}
  \label{fig:meancompare_conf2}
\end{figure}

\subsection*{Assets}
I then explore the relationship between asset holdings and GBV. The first asset I consider, is having a tin roof (figure \ref{fig:meancompare_ses1}). These roofs are a substantial improvement over thatch roofs, but about half the sample doesn't own them. This makes for a simple, yet non-arbitrary, way to split the sample in richer and poorer households.   In the group without tin roofs (n=\incid{tinroof0}{n}), the average number of issues was \incid{tinroof0}{mean0} for the control group, and \incid{tinroof0}{mean1} for the treatment group. This implies a rate of GBV of \incid{tinroof0}{incidence_pct}\% (p = \incid{tinroof0}{p}). For the group with tin roofs (n=\incid{tinroof1}{n}), the average number of issues was \incid{tinroof1}{mean0} for the control group, and \incid{tinroof1}{mean1} for the treatment group. This implies a rate of GBV of \incid{tinroof1}{incidence_pct}\% (p = \incid{tinroof1}{p}).

\begin{figure}[hp]
  \includegraphics[width=0.6\linewidth]{\figureloc/meancompare_ses1.png}
  \caption{Comparison of means of issues faced across asset holdings (tin roof).}
  \label{fig:meancompare_ses1}
\end{figure}

\paragraph{}
A second asset with important status implications for the household is livestock (see figure \ref{fig:meancompare_ses2}).  In the group without livestock (n=\incid{livestockany0}{n}), the average number of issues was \incid{livestockany0}{mean0} for the control group, and \incid{livestockany0}{mean1} for the treatment group. This implies a rate of GBV of \incid{livestockany0}{incidence_pct}\% (p = \incid{livestockany0}{p}). For the group with tin roofs (n=\incid{livestockany1}{n}), the average number of issues was \incid{livestockany1}{mean0} for the control group, and \incid{livestockany1}{mean1} for the treatment group. This implies a rate of GBV of \incid{livestockany1}{incidence_pct}\% (p = \incid{livestockany1}{p}).

\begin{figure}[hp]
\includegraphics[width=0.6\linewidth]{\figureloc/meancompare_ses2.png}
  \caption{Comparison of means of issues faced across asset holdings (livestock).}
  \label{fig:meancompare_ses2}
\end{figure}

\paragraph{}
In summary, I find that women who are in a disadvantageous position in their household, either because they are of lower status, or because they have less bargaining power, are more likely to have experienced GBV. The results with respect to conflict were ambiguous, and I find no suggestion that respondents in poorer household experience GBV at a higher rate than their peers in richer households. If anything, I find the opposite. There are two large caveats with these findings: (i) they are not causal; (ii) by running separate comparisons, any conclusions are at risk of missing variable bias. While the first one is hard to address using list experiments, it is possible to come to more rigourous estimates of effects than presented so far. 

\subsection*{Regression analysis}
\paragraph{}
In table \ref{tab:results_regression} I display the results of a number of regressions. These are linear regressions: the coefficients reported should be interpreted as the marginal effects on the difference between control and treatment (i.e. the interaction between the variable and the treatment indicator). For brevity, the level effects of each variable on the total number of issues is omitted, as these coefficients are not of interest here. In the first column, I report the results of including only endline survey measures; in the second, I add baseline conflict indicators; in the third I include results from the bargaining game. Finally, I estimate a full model. I find that neither conflict exposure, nor household asset holdings are correlated to the incidence of GBV. Intra-household dynamics on the other hand, are strongly correlated to GBV. Nearly all the GBV happens to women who live in male dominated households. From this, we cannot conclude that it is the husbands who perpetrate the violence. It may be possible that both GBV, and choosing partners of high status, are both determined by the same factors. However, contrary to popular narratives, we find no evidence of correlation to conflict exposure.


\newcommand{\coeffget}[3]{\csvreader[filter=\equal{\reg}{#1} \and \equal{\var}{#2}]{\tableloc/regs.csv}{var=\var,reg=\reg,#3=\coeff}{\coeff}}
%TEST: \coeffget{l1}{Delta:husbmoreland}{coef}

\begin{table}
	\caption{Results}\label{tab:results_regression}
	\begin{center}
	\input{"C:/Users/Koen/Dropbox/PhD/Papers/CongoGBV/Tables/results_regression.tex"}
	\end{center}
\end{table}


\section*{Conclusion}
%restate the topic and its importance.
This study provided evidence on the victims of SGBV in Congo. Prevalence of SGBV is high in Congo, but little is known about the victims. To obtain this evidence, I used a list experiment to prevent disclosure bias.This allows me to find correlates of SGBV. I complemented this with a household bargaining game, designed to elicit often unobserved intra-household dynamics.

\paragraph{} 
In contrast to popular frames that paint a picture of conflict being the largest determinant of sexual violence (and sexual violence the most important outcome of conflict), I find no relationship between conflict exposure and SGBV. I also find no evidence for a difference in incidence of SGBV between poorer and richer household. I do find that victims of sexual violence were married to men of higher status (pre-marriage), and had less bargaining power in the household.

\paragraph{} 
These findings correspond to previous literature suggesting that intimate partners are more likely perpetrators of SGBV than members of armed groups \citep[see e.g.][]{Peterman2011}. The implication is that an end to the conflict in Eastern Congo will not bring an end to the problems women face. To reduce the incidence of SGBV, strong efforts to promote female empowerment are needed.







%interpretatie voorbeelden:
%\cite{Meng2017} first simply present mean differences; they then present a maximum likelihood model. Based on this model, they predict mean differences and present this as their quantity of interest. They have replication files online.

%\cite{Imai2011}: himself admits that the coefficients of the maximum likelihood method are difficult to compare. He therefore generates predicted values of shares of people answering affirmitively to the sensitive item for two subgroups.

%\cite{Bulte2019}: how they do interpret.


%bibliography, this is needed for bibtex
\clearpage 
\bibliographystyle{chicago}
%path to .bib file (e.g. automatically exported by mendeley) exclude the file extension!
\bibliography{C:/Users/Koen/Dropbox/Literatuur/Mendeley/Bibtex/CongoGBV}

\end{document}
