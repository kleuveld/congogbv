\documentclass[11pt,a4paper]{scrartcl} %scrartcl from KOMA-script uses sans headers and looks less latexy than article
%\usepackage[doublespacing]{setspace}
%\usepackage[capposition=top]{floatrow} %for notes under pics
%\usepackage[nomarkers]{endfloat}
\usepackage{natbib}
\usepackage[utf8]{inputenc}
\usepackage{amsmath}
\usepackage{amsfonts}
\usepackage{amssymb}
\usepackage[left=2.5cm,right=2.5cm,top=3cm,bottom=3cm]{geometry}
\usepackage[pdftex, hidelinks]{hyperref}
\usepackage{graphicx}
\usepackage{booktabs}
\usepackage[blocks]{authblk}
\usepackage{url}
\usepackage{tabularx}
\usepackage{enumitem}
\setlist{noitemsep} % \setlist{nosep} to reove separation around lists as well


%citation cheat sheet:
	%\cite{key}					Jones et al. (1990)
	% \citet{key}				Jones et al. (1990)
	% \citet*{key}				Jones, Baker, and Smith (1990)
	% \citep{key}				(Jones et al. 1990)
	% \citep*{key}				(Jones, Baker, and Smith 1990)
	% \citep[p.~99]{key}		(Jones et al., 1990, p. 99)
	% \citep[e.g.][]{key}		(e.g. Jones et al., 1990)
	% \citep[e.g.][p.~99]{key}	(e.g. Jones et al., 1990, p. 99)
	% \citeauthor{key}			Jones et al.
	% \citeauthor*{key}			Jones, Baker, and Smith
	% \citeyear{key}			1990
	% \citealt{key} 			Jones et al. 1990

\begin{document}
\author{Koen Leuveld}

\affil{Development Economics Group, Wageningen University \\
\href{mailto:koen.leuveld@wur.nl}{koen.leuveld@wur.nl}}



\title{Sexual Violence in Eastern DRC}
\subtitle{Evidence from a list experiment} %needs scrartcl

\maketitle

%\begin{abstract}
%Abstract to come
%\end{abstract}


%%%%%%%%%%%%%%%%%%%%%%%%%%
\section*{Introduction}
%%%%%%%%%%%%%%%%%%%%%%%%%

\paragraph{}
%Hook
Eastern DR Congo has been called the "rape capital of the world". Much of this has been linked to the decades of violence that the region has seen over the past decades. Various armed groups have used Gender Based Violence (GBV) as a weapon against the population. This has put the issue on the radar of the international community. Today, most of the NGOs operating in the region focus on GBV, either exclusively, or as part of broader programmes. The 2018 Nobel Peace prize was awarded to Dr. Denis Mukwege, for his work on victims of GBV at Bukavu's Panzi Hospital.

\paragraph{}
%Research question
The issue of GBV is complicated. While the armed conflict is universally seen as a large driver of GBV, customs and tradition leave Congolese women in a powerless position. This paper aims to provide quantitative evidence on the correlates of sexual violence.

\paragraph{}
%Antecedents
Antecedents of this paper include (perhaps weave this in with contribution):
\begin{itemize}
	\item Conflict literature on Congo:
	\item GBV literature:
%\cite{Hilhorst2018}:  question whether the reponse of the international community should be labelled a hype.
%\cite{Quattrochi2019}: "Empowerment programs often target women who have survived sexual and gender-basedviolence (SGBV), with the justification that these women may develop disempowered beliefs as a copingmechanism, or face greater barriers to, or derive greater benefits from, the adoption of empowered beliefs andpreferences."
%\cite{Porter2019}: argues that rape as a weapon of war is not helpful. It is embedded in existing mores and customs. 
%close to my paper: \cite{Stark2017} assess disclosement bias.
	\item List experiment literature: 
%Classic example: The 1991 National Race and Politics Survey (\cite{Sniderman1991}). Used to measure racial prejudice. Recently, more of of them have been done. \cite{Imai2011}: introduction of new estimator. \cite{Blair2012}: overview of literature. \cite{Tsai2019}: implement methods in Stata.
\end{itemize}

\paragraph{}
%Contribution
While from the literature it is obvious that GBV is endemic in Eastern Congo, it is difficult to get quantitative evidence on the problem. The area is poorly accessible to research teams, and women may be unwilling to divulge information on victimization due to the stigma involved. We collected data in the region in 2012, as part of the evaluation of Dutch development efforts. The questionnaires included a section on gender. As part of this section, a list experiment was administered.

\paragraph{}
%Road Map


%%%%%%%%%%%%%%%%%%%%%%%%%%%%%%%%
\section*{Background and data}
%%%%%%%%%%%%%%%%%%%%%%%%%%%%%%%%
\paragraph{}
Data for this research comes from an evaluation of Dutch development aid. A household survey formed part of this evaluation. The final module of this survey dealt with gender issues, and was administered to an adult female in the household.  To collect data on the sensitive topic of GBV, this module contained a list experiment. For this, the respondents were randomly divided into two groups. Each group was presented with a number of problems that women can realistically face in Eastern Congo. They then indicated how many of the problems applied to them. The first group (Group 0) was presented with the following issues:
\begin{itemize}
	\item Lack of food;
	\item Lack of money;
	\item Theft; and,
	\item Sterility.
\end{itemize}

\paragraph{}
The women then indicated how many of these issues were applicable to them. Women in the second group (Group 1) were presented with the same four items, but a fifth item was added: Sexual Violence. This means we cannot know who among our respondents was the victim of GBV. All we know for the individual respondents is the number of problems. However, since any difference in the two groups will be caused by the addition of the fifth item, a simple comparison of the mean number of issues faced by women in both groups reveals the population-level incidence of GBV. However, my interest here goes beyond the computation of an incidence rate. The goals is to find correlates of GBV. To do so, note that the computation of the diffence in means requires nothing esle than to regress the number of issues faced on the treatment indicator. To include more covariates, I simply interact them with the treatment indicator.

\paragraph{}
\cite{Imai2011} identifies two issues with this linear estimation technique. First, the predicted values it generates for the probability of answering affirmitavily to being a victim of GBV can lie outside the 0-1 range. Second, the method is inefficient. He proposes a more efficient Maximum likelihood estimator. However, given the inherent difficulties in interpreting the coefficients obtained through his method, here I prefer to present the results of the linear model first. I present results from the ML model as robustness checks. 


Table \ref{table:balance} presents descriptive statistics of the sample. Our sample consists of 508 married women. Their a

\begin{table}
	\label{table:balance}
	\begin{center}
	\input{"C:/Users/Koen/Dropbox (Personal)/PhD/Papers/CongoGBV/Tables/balance.tex"}
	\end{center}
\end{table}





%some examples on how to interpret list experiments:
%\cite{Meng2017} run various types of models. Consistently refer to mean difference models, and Maximum Likelihood models

%\cite{Blair2014}: extremely detailed stuff. Thread carefully!
%\cite{Frye2017} simple difference in means model.

%%%%%%%%%%%%%%%%%%%%%%%%%%%%%%%%
\section*{Results}
%%%%%%%%%%%%%%%%%%%%%%%%%%%%%%%%
\paragraph{}
See the results in Table \ref{table:results}. Columns 1 and 2 present the results from a simple linear model. For brevity, the table only reports the delta coefficients of interest, not the sigma and psi ones. The constant in column 1 means that 26 percent of the women in our sample indicate ever having been victim of sexual violence. 
I then include the following correlates:
\begin{itemize}
	\item age: the age of the respondent in years.
	\item victim\_any: 1 if the household was victimized by the conflict (in terms of property damage, loss of assest, or loss of life.)
	\item dot\_husband: the amount of money the husband and his family contributed to the marriage. This includes the bride price, as well as contributions to the festivities.
	\item dot\_wife: the amount of money  the wife and here family contributed to the marriage. This will typically consist of contributions to the festivities.
	\item mar\_rap: indicates a forced marriage.
	\item mar\_agediff: the age the husband minus the age of the wife.
	\item terrfe2 and terrfe3: territoire fixed effects.
\end{itemize}

Note that including these marriage indicators, significantly alters our sample size, as any unmarried women drop out.

\paragraph{}
We find that women from victimized households are less likely to be victimized by GBV. On the contrary, forced marriages and age differences between husband and wife are positively and significantly correlated to marriage indicators. I then also run the model using Imai's estimate, but completely fail to make sense of the coefficients. (it's based on logit.)


%interpretatie:
%\cite{Meng2017} first simply present mean differences; they then present a maximum likelihood model. Based on this model, they predict mean differences and present this as their quantity of interest. They have replication files online.

%\cite{Imai2011}: himself admits that the coefficients of the maximum likelihood method are difficult to compare. He therefore generates predicted values of shares of people answering affirmitively to the sensitive item for two subgroups.


\begin{table}
	\caption{Results}\label{table:results}
	\begin{center}
	\input{"C:/Users/Koen/Dropbox (Personal)/PhD/Papers/CongoGBV/Tables/results.tex"}
	\end{center}
\end{table}



%bibliography, this is needed for bibtex
\clearpage 
\bibliographystyle{chicago}

%path to .bib file (e.g. automatically exported by mendeley) exclude the file extension!
\bibliography{"C:/Users/Koen/Dropbox (Personal)/Literatuur/Mendeley/Bibtex/Congo GBV"}
\end{document}
